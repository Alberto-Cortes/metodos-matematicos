\documentclass{article}
\usepackage{amsmath, amssymb, amsthm}
\usepackage[utf8]{inputenc}
\usepackage{geometry}
\geometry{
a4paper,
total={170mm,257mm},
left=25mm,
top=20mm,
}

\newcommand{\norm}[1]{\left\lVert#1\right\rVert}
\newcommand{\custvec}[1]{\left\langle#1\right\rangle}

\title{Métodos Matemáticos}
\author{Alberto Cortés González - A01635875}
\date{Agosto 2020}

\begin{document}

\maketitle

\section{Actividad 1}
\begin{proof}
$\norm{a\times b} = \norm{a}\norm{b}\sin\theta \\$
A continuaciónn vamos a demostrar que la norma del resultado del producto cruz entre 2 vectores $a$ y $b$ es equivalente al producto entre la norma de cada uno de los vectores por el seno del ángulo entre ambos.\\

Comenzamos elevando nuestro término de la izquierda al cuadrado, evaluamos el producto cruz y desarrollamos completamente los términos.
\begin{equation*}
    \begin{split}
        \norm{a\times b}^2  &= (a_{2}b_{3} - a_{3}b_{2})^2 + (a_{3}b_{1} - a_{1}b_{3})^2 + (a_{1}b_{2} - a_{2}b_{1})^2\\
        (a_{2}b_{3} - a_{3}b_{2})^2 &= (a_{2}b_{3})^2 - 2(a_{2}b_{2}a_{3}b_{3}) + (a_{3}b_{2})^2 \\
        (a_{3}b_{1} - a_{1}b_{3})^2 &= (a_{3}b_{1})^2 - 2(a_{3}b_{1}a_{1}b_{3}) + (a_{1}b_{3})^2 \\
        (a_{1}b_{2} - a_{2}b_{1})^2 &= (a_{1}b_{2})^2 - 2(a_{1}b_{2}a_{2}b_{1}) + (a_{2}b_{1})^2 \\
        \norm{a\times b}^2  &= a_{2}^2b_{3}^2 - 2a_{2}b_{2}a_{3}b_{3} + a_{3}^2b_{2}^2 \\
        &+ a_{3}^2b_{1}^2 - 2a_{1}b_{1}a_{3}b_{3} + a_{1}^2b_{3}^2 \\
        &+ a_{1}^2b_{2}^2 - 2a_{1}b_{1}a_{2}b_{2} + a_{2}^2b_{1}^2 \\
        &= a_{1}^2(b_{2}^2+b_{3}^2)+a_{2}^2(b_{1}^2+b_{3}^2)+a_{3}^2(b_{1}^2+b_{2}^2)\\
        &- 2(a_{2}b_{2}a_{3}b_{3}+a_{1}b_{1}a_{3}b_{3}+a_{1}b_{1}a_{2}b_{2}) \\
    \end{split}
\end{equation*}

Habiendo terminado ese desarrollo comenzamos uno nuevo, esta vez tomaremos el cuadrado de la norma de $a$ por el cuadrado de la norma de $b$ por el coseno cuadrado de teta. Esto siendo equivalente al cuadrado del pruducto punto entre ambos vectores y buscando llegar más adelante al lado derecho de nuestro plantemiento inicial.
\begin{equation*}
    \begin{split}
            \norm{a}^2\norm{b}^2\cos^2\theta=(a\cdot b)^2   &= (a_{1}b_{1}+a_{2}b_{2}+a_{3}b_{3})^2\\
                &= a_{1}^2b_{1}^2+a_{2}^2b_{2}^2+a_{3}^2b_{3}^2 \\
                &+ 2(a_{1}b_{1}a_{2}b_{2}+a_{1}b_{1}a_{3}b_{3}+a_{2}b_{2}a_{3}b_{3}) \\
    \end{split}
\end{equation*}

Ahora sumamos los resultados de los 2 desarrollos anteriores en la parte derecha de nuestra ecuación.
\begin{equation*}
    \begin{split}
            \norm{a\times b}^2 + \norm{a}^2\norm{b}^2\cos^2\theta   &= a_{1}^2(b_{2}^2+b_{3}^2)+a_{2}^2(b_{1}^2+b_{3}^2)+ a_{3}^2(b_{1}^2+b_{2}^2) \\
            &+ a_{1}^2b_{1}^2+a_{2}^2b_{2}^2+a_{3}^2b_{3}^2 \\
            &= a_{1}^2(b_{1}^2+b_{2}^2+b_{3}^2)+a_{2}^2(b_{1}^2+b_{2}^2+b_{3}^2)+ a_{3}^2(b_{1}^2+b_{2}^2+b_{3}^2) \\
            &= (a_{1}^2+a_{2}^2+a_{3}^2)(b_{1}^2+b_{2}^2+b_{3}^2) \\
            &= \norm{a}^2\norm{b}^2
    \end{split}
\end{equation*}

Finalmente restamos el segundo desarrollo a ambos lados y llegamos a lo que buscamos demostrar.\\
\begin{equation*}
    \begin{split}
        \norm{a\times b}^2 &= \norm{a}^2\norm{b}^2 - \norm{a}^2\norm{b}^2\cos^2\theta \\
        &= \norm{a}^2\norm{b}^2(1-\cos^2\theta) \\
        &= \norm{a}^2\norm{b}^2\sin^2\theta \\
        & \therefore \norm{a\times b} = \norm{a}\norm{b}\sin\theta 
    \end{split}
\end{equation*}
\end{proof}
\newpage
\section{Actividad 2}
Dados los vectores $\vec{a}=\custvec{1,1,0}$, $\vec{b}=\custvec{0,1,0}$ y $\vec{c}=\custvec{1,1,1}$ encuentraremos los vectores recíprocos para cada uno de estos.\\

Comenzaremos por calcular $\vec{a}\cdot (\vec{b}\times \vec{c})$ pues este valor será utilizado a lo largo del cálculo de los 3 vectores recíprocos como el denominador de cada uno.\\
\begin{equation*}
    \begin{split}
        (\vec{b}\times \vec{c}) &= \begin{vmatrix}
        \hat{i} & \hat{j} & \hat{k} \\
        0 & 1 & 0 \\
        1 & 1 & 1
        \end{vmatrix} = \hat{i} - \hat{k} = \custvec{1,0,-1}\\
        \vec{a}\cdot (\vec{b}\times \vec{c}) &= \custvec{1,1,0} \cdot \custvec{1,0,-1} = 1
    \end{split}
\end{equation*}

Teniendo este valor podemos seguir adelante y calcular el producto cruz respectivo para encontrar el vector recíproco de cada uno de nuestros vectores, comenzando por el vector $a$.\\
\begin{equation*}
    \begin{split}
        \vec{a'} = \frac{(\vec{b}\times \vec{c})}{\vec{a}\cdot (\vec{b}\times \vec{c})} &= 1\cdot \begin{vmatrix}
        \hat{i} & \hat{j} & \hat{k} \\
        0 & 1 & 0 \\
        1 & 1 & 1
        \end{vmatrix} = \hat{i} - \hat{k} = \custvec{1,0,-1}\\
        \vec{a}\cdot \vec{a'} &= \custvec{1,1,0}\cdot \custvec{1,0,-1} = 1 
    \end{split}
\end{equation*}

Ahora calcularemos el vector recíproco de $b$.
\begin{equation*}
    \begin{split}
        \vec{b'} = \frac{(\vec{c}\times \vec{a})}{\vec{a}\cdot (\vec{b}\times \vec{c})} &= 1\cdot \begin{vmatrix}
        \hat{i} & \hat{j} & \hat{k} \\
        1 & 1 & 1 \\
        1 & 1 & 0
        \end{vmatrix} = \hat{j} - \hat{i} = \custvec{-1,1,0}\\
        \vec{b}\cdot \vec{b'} &= \custvec{0,1,0}\cdot \custvec{-1,1,0} = 1 
    \end{split}
\end{equation*}

Finalmente calculamos el vector recíproco de $c$.
\begin{equation*}
    \begin{split}
        \vec{c'} = \frac{(\vec{a}\times \vec{b})}{\vec{a}\cdot (\vec{b}\times \vec{c})} &= 1\cdot \begin{vmatrix}
        \hat{i} & \hat{j} & \hat{k} \\
        0 & 1 & 0 \\
        1 & 1 & 1
        \end{vmatrix} = \hat{k} = \custvec{0,0,1}\\
        \vec{c}\cdot \vec{c'} &= \custvec{1,1,1}\cdot \custvec{0,0,1} = 1 
    \end{split}
\end{equation*}
\newpage
\section{Actividad 3}
Evalua la siguiente suma con la delta de Kronecker.\\
\begin{equation*}
    \begin{split}
        \sum\limits_{k=1}^{10} \delta_{i,k}\delta_{k,j}\quad(1\leqslant i,j \leqslant 10)\\
    \end{split}
\end{equation*}
Algo que podemos notar es que en cada iteración de la sumatoria tendremos la multiplicación de ambas deltas, cada una con sus respectivos subíndices. Sabemos que cuando los índices del delta sean distintos este dará 0, siguiendo esto cualquier combinación de los 3 subíndices resultaría en uno de 4 casos, $0\cdot 0$, $0\cdot 1$, $1\cdot 0$ o $1\cdot1$, siendo este último el único que nos interesa.\\

Notando que este último solo se da cuando $i=k$ y $k=j$, entonces necesitamos que $i=j=k$ para tener un valor distinto a 0, reduciendo las 1000 posibles combinaciones a solo 10 que aportan a la suma.\\
\begin{equation*}
    \begin{split}
        \sum\limits_{k=1}^{10}\delta_{i,j}\quad (1\leqslant i,j \leqslant 10) = \sum\limits_{k=1}^{10}1 = 10\\
    \end{split}
\end{equation*}
\newpage
\section{Actividad 4}
Descripción de las propiedades de la Delta de Kronecker y el Símbolo de Levi-Civita asó como su relación.
\begin{itemize}
    \item $\epsilon_{ijk}=\epsilon_{jki}=\epsilon_{kij}$
    \begin{itemize}
        \item La primera propiedad del Símbolo de Levi-Civita dice que todas las permutaciones cíclicas 
        de sus índices serán iguales, es decir, tendran el valor de 1.
    \end{itemize}
    \item $\epsilon_{ijk}=-\epsilon_{ikj}$
    \begin{itemize}
        \item La segunda propiedad dice que una permutación cíclica de los índices es igual el negativo 
        de una permutación anticíclica. Esto siendo verdadero puesto que $1=-(-1)$.
    \end{itemize}
    \item $\epsilon_{ijk}\epsilon_{imn}=\delta_{jm}\delta_{kn}-\delta_{jn}\delta_{km}$
    \begin{itemize}
        \item La tercera propiedad surge de expresar el producto entre dos $\epsilon$ que cuentan con un
        índice repetido como el determinante de una matriz con la combinatoria de índices para $\delta$, obteniendo el determinante
        como resultado del producto y reduciendo algebraicamente es que llegamos a este resultado.
    \end{itemize}
    \item $\epsilon_{ijk}\epsilon_{ijn}=2\delta_{kn}$
    \begin{itemize}
        \item La cuarta propiedad surge del producto de dos símbolos de $\epsilon$ que comparten dos de sus índices,
        en este caso observando que tendremos más términos similares y es por eso que la reducción será mayor, llegando
        a un solo término.
    \end{itemize}
    \item $\epsilon_{ijk}\epsilon_{ijk}=6$
    \begin{itemize}
        \item La quinta propiedad sigue los pasos de las 2 anteriores, ahora siendo el producto de dos $\epsilon$ que comparten
        sus tres índices, reduciendo a un único término $2\delta_{ii}$ con $\delta_{ii}=3$ por definición, llegamos a un simple $2\cdot 3=6$.
    \end{itemize}
    \item $\epsilon_{ijk}=\frac{1}{2}(i-j)(j-k)(k-i)$, para $(i,j,k)=1,2,3$
    \begin{itemize}
        \item Finalmente tenemos la sexta propiedad, esta sirve para demostrar el valor de $1$ de la misma usando un caso de ejemplo con los
        valores descritos para cada índice. Sustituyendo directemente en la ecuación tendríamos lo siguiente.\\
        \begin{equation*}
            \begin{split}
                \epsilon_{ijk}&=\frac{1}{2}(1-2)(2-3)(3-1)\\
                &=\frac{1}{2}(-1)(-1)(2)\\
                &=\frac{1}{2}(1)(2)\\
                &=1
            \end{split}
        \end{equation*}
    \end{itemize}
\end{itemize}
\newpage
\section{Actividad 5}
Para una partícula moviéndose en una orbita circular $\pmb{r} = r\cos(\omega t)\pmb{\hat{x}}+ r\sin(\omega t)\pmb{\hat{y}}$:
\begin{enumerate}
\item Evalúa $\pmb{r}\times \pmb{\dot{r}}$, $\pmb{\dot{r}}$ siendo $= \frac{d\pmb{r}}{dt}=\pmb{v}$.\\
    Comenzamos por evaluar $\pmb{\dot{r}}$, derivando con respecto a $t$.
    \begin{equation*}
        \begin{split}
        \pmb{\dot{r}}=\frac{d}{dt}r\cos(\omega t)\pmb{\hat{x}} + \frac{d}{dt}r\sin(\omega t)\pmb{\hat{y}}=-r\omega\sin(\omega t)\pmb{\hat{x}}+r\omega\cos(\omega t)\pmb{\hat{y}}=\pmb{v}
        \end{split}
    \end{equation*}
    Seguido de esto evaluamos el producto cruz entre $\pmb{r}$ y $\pmb{\dot{r}}$, este al no estar definido en ${\rm I\!R}^2$ añadiremos una componente en $\pmb{\hat{z}}$ con valor de $0$ para ambos vectores.
    \begin{equation*}
        \begin{split}
            \pmb{r}\times \pmb{\dot{r}}&= \begin{vmatrix}
            \pmb{\hat{x}} & \pmb{\hat{y}} & \pmb{\hat{z}} \\
            r\cos(\omega t) & r\sin(\omega t) & 0 \\
            -r\omega\sin(\omega t) & r\omega\cos(\omega t) & 0
            \end{vmatrix}\\
            &=\pmb{\hat{x}}\begin{vmatrix}
            r\sin(\omega t) & 0 \\
            r\omega\cos(\omega t) & 0
            \end{vmatrix}-\pmb{\hat{y}}\begin{vmatrix}
            r\cos(\omega t) & 0\\
            -r\omega\sin(\omega t) & 0
            \end{vmatrix}+\pmb{\hat{z}}\begin{vmatrix}
            r\cos(\omega t) & r\sin(\omega t)\\
            -r\omega\sin(\omega t) & r\omega\cos(\omega t)
            \end{vmatrix}\\
            &=(r^2\omega\cos^2(\omega t) + r^2\omega\sin^2(\omega t))\pmb{\hat{z}}\\
            &=(r^2\omega(\cos^2(\omega t)+\sin^2(\omega t)))\pmb{\hat{z}}\\
            &=r^2\omega\pmb{\hat{z}}
        \end{split}
    \end{equation*}
    \item Muestra que $\pmb{\ddot{r}}+\omega^2 \pmb{r}=0$ con $\pmb{\ddot{r}}=\frac{d\pmb{v}}{dt}$.\\
    \begin{proof}
    Partiremos suponiendo que  $\pmb{\ddot{r}}+\omega^2 \pmb{r}=0$.\\
    Primero obtendremos $\pmb{\ddot{r}}$.
    \begin{equation*}
        \begin{split}
            \pmb{\ddot{r}}&=\frac{d^2}{dt^2}r\cos(\omega t)\pmb{\hat{x}} + \frac{d^2}{dt^2}r\sin(\omega t)\pmb{\hat{y}}=-r\omega^2\cos(\omega t)\pmb{\hat{x}}-r\omega^2\sin(\omega t)\pmb{\hat{y}}\\
            &=-\omega^2(r\cos(\omega t)\pmb{\hat{x}}+r\sin(\omega t)\pmb{\hat{y}})
        \end{split}
    \end{equation*}
    Ahora que ya tenemos nuestro vector podemos realizar la suma igualada a $0$.
    \begin{equation*}
        \begin{split}
            -\omega^2(r\cos(\omega t)\pmb{\hat{x}}+r\sin(\omega t)\pmb{\hat{y}})+\omega^2r=0\\
            \omega^2r=\omega^2(r\cos(\omega t)\pmb{\hat{x}}+r\sin(\omega t)\pmb{\hat{y}})\\
        \end{split}
    \end{equation*}
    Buscando igualar ambos lados obtendremos la norma de ellos.
    \begin{equation*}
        \begin{split}
            \norm{\omega^2r}&=\norm{\omega^2}\norm{(r\cos(\omega t)\pmb{\hat{x}}+r\sin(\omega t)\pmb{\hat{y}})}\\
            \omega^2r&=\omega^2\sqrt{r^2\cos^2(\omega t)+r^2\sin^2(\omega t)}\\
            \omega^2r&=\omega^2\sqrt{r^2(\cos^2(\omega t)+\sin^2(\omega t))}\\
            \omega^2r&=\omega^2\sqrt{r^2}\\
            \omega^2r&=\omega^2r\\
        \end{split}
    \end{equation*}
    Al ser esto cierto podemos asumir que nuestra suposición inicial era correcta.
    \begin{equation*}
        \begin{split}
            \therefore \pmb{\ddot{r}}&+\omega^2 \pmb{r}=0
        \end{split}
    \end{equation*}
    \end{proof}
    \begin{itemize}
        \item El radio $\pmb{r}$ y la velocidad angular $\omega$ son constantes.
    \end{itemize}
\end{enumerate}
\newpage
\section{Actividad 6}
Calcula la curvatura de $r(t)=\custvec{t^2,\sin (t) - t\cos (t), \cos (t) + t\sin (t)}$, donde $t > 0$.\\

Para obtener la curvatura usaremos el teorema que dicta lo siguiente:\\
\begin{equation*}
    \kappa(t) = \frac{\norm{T'}}{\norm{r'}}=\frac{\norm{r' \times r''}}{\norm{r'}^3}
\end{equation*}

Lo primero que haremos será obtener tanto $r'$ como $r''$, pues calcularemos la curvatura a partir de estas.\\
\begin{equation*}
    \begin{split}
        \frac{d}{dt}r=r'&=\custvec{\frac{d}{dt}t^2,\frac{d}{dt}(\sin (t) - t\cos (t)),\frac{d}{dt}(\cos (t) + t\sin (t))}\\
        &=\custvec{2t, t\sin (t), t\cos (t)}\\ \\
        \frac{d^2}{dt^2}r=r''&=\custvec{\frac{d^2}{dt^2}t^2,\frac{d^2}{dt^2}(\sin (t) - t\cos (t)),\frac{d^2}{dt^2}(\cos (t) + t\sin (t))}\\
        &=\custvec{2, t\cos (t) + \sin (t), \cos (t) - t\sin (t)}
    \end{split}
\end{equation*}

Seguido de esto calculamos el numerador de nuestra ecuación antes de obtener su norma.
\begin{equation*}
    \begin{split}
        r' \times r''&= \begin{vmatrix}
                \hat{i} & \hat{j} & \hat{k} \\
                2t & t\sin (t) & t\cos (t) \\
                2 & t\cos (t) + \sin (t) & \cos (t) - t\sin (t)
            \end{vmatrix}\\
            &=\hat{i}\begin{vmatrix}
                t\sin (t) & t\cos (t) \\
                t\cos (t) + \sin (t) & \cos (t) - t\sin (t)
            \end{vmatrix}\\
            &-\hat{j}\begin{vmatrix}
                2t & t \cos (t) \\
                2 & \cos (t)-t\sin (t)
            \end{vmatrix}\\
            &+\hat{k}\begin{vmatrix}
                2t & t\sin (t) \\
                2 & t\cos (t) + \sin (t)
            \end{vmatrix} \\
            &=-t^2\hat{i}+2t^2\sin (t)\hat{j}+2t^2\cos (t)\hat{k}
    \end{split}
\end{equation*}

Ahora que ya lo tenemos podemos calcular su norma.
\begin{equation*}
    \begin{split}
        \norm{r' \times r''}&=\norm{\custvec{-t^2,2t^2\sin (t), 2t^2\cos (t)}} \\
        &=\sqrt{(-t^2)^2+(2t^2\sin  (t))^2+(2t^2\cos (t))^2} \\
        &=\sqrt{t^4+4t^4\sin^2(t)+4t^4\cos^2(t)} \\
        &=\sqrt{t^4+4t^4(\sin^2(t)+\cos^2(t))} \\
        &=\sqrt{5t^4}=\sqrt{5}t^2
    \end{split}
\end{equation*}

Una vez que obtuvimos el numerador encontraremos el denominador.
\begin{equation*}
    \begin{split}
        \norm{r'}^3&=\norm{\custvec{2t, t\sin (t), t\cos (t)}}^3 \\
        &=\sqrt{(2t)^2+(t\sin (t))^2+(t\cos (t))^2}^3 \\
        &=\sqrt{4t^2+t^2\sin^2(t)+t^2\cos^2(t)}^3 \\
        &=\sqrt{4t^2+t^2(\sin^2(t)+\cos^2(t))}^3 \\
        &=\sqrt{5t^2}^3=5\sqrt{5}t^3
    \end{split}
\end{equation*}

Finalmente resolvemos para obtener la curvatura.
\begin{equation*}
    \begin{split}
        \kappa(t) = \frac{\sqrt{5}t^2}{5\sqrt{5}t^3} = \frac{1}{5t}
    \end{split}
\end{equation*}
\newpage
\section{Actividad 7}
Determine el trabajo efectuado por el campo de fuerza $\pmb{F}(x,y)=x^2\pmb{\hat{i}}
-xy\pmb{\hat{j}}$ cuando se mueve una partícula a lo largo del cuarto de
circunferencia $\pmb{r}(t)=\cos(t)\pmb{\hat{i}}+\sin(t)\pmb{\hat{j}}, 
0 \leqslant t \leqslant \frac{\pi}{2}.$\\

Comenzamos por obtener la primera derivada de $\pmb{r}$, esta siendo $\pmb{r'}$.\\
\begin{equation*}
    \pmb{r'}=\frac{d}{dt}\pmb{r}=\frac{d}{dt}\cos(t)\pmb{\hat{i}}+\frac{d}{dt}\sin(t)
    \pmb{\hat{j}}=-\sin(t)\pmb{\hat{i}}+\cos(t)\pmb{\hat{j}}
\end{equation*}

Posteriormente evaluamos este nuevo vector en el campo de fuerza.
\begin{equation*}
    \pmb{F}(\pmb{r}(t))=\cos^2(t)\pmb{\hat{i}}-\cos(t)\sin(t)\pmb{\hat{j}}
\end{equation*}

Siguiendo la definición de la integral de línea en campos vectoriales podemos realizar
nuestro cálculo directamente.
\begin{equation*}
    \begin{split}
        &\int_{C}\pmb{F}\cdot d\pmb{r}=\int_{a}^{b}\pmb{F}(\pmb{r}(t))\cdot \pmb{r'}
    (t)dt =\int_{C}\pmb{F}\cdot\pmb{T}ds\\
        &\int_{0}^{\frac{\pi}{2}}(\cos^2(t)\pmb{\hat{i}}-\cos(t)\sin(t)\pmb{\hat{j}})
        \cdot(-\sin(t)\pmb{\hat{i}}+\cos(t)\pmb{\hat{j}})dt\\
        &\int_{0}^{\frac{\pi}{2}}-2(\sin(t)\cos^2(t))dt \qquad\qquad u=\cos(t)\\
        &\int_{0}^{\frac{\pi}{2}}-2u^2du=\frac{2}{3}u^3\Big|_{0}^{\frac{\pi}{2}}=0-\frac{2}{3}=-\frac{2}{3}
    \end{split}
\end{equation*}
\newpage
\section{Actividad 8}
Si $\pmb{F}(x,y,z)=y^2\pmb{\hat{i}}+(2xy+e^{3z})\pmb{\hat{j}}+3ye^{3z}\pmb{\hat{k}}$, determine una función 
$f(x,y,z)$ tal que $\nabla f(x,y,z) = \pmb{F}(x,y,z)$.\\

Comenzamos separando nuestro campo en sus derivadas parciales correspondientes para cada variable $x,y,z$. 
Asumiendo que es un campo conservativo estas se obtienen directamente como el término de cada dirección 
$\pmb{\hat{i}},\pmb{\hat{j}},\pmb{\hat{k}}$, las llamaremos $P,Q,R$.
\begin{equation*}
    P=f_{x}(x,y,z)=y^2 \qquad Q=f_{y}(x,y,z)=2xy+e^{3z} \qquad R=f_{z}(x,y,z)=3ye^{3z}
\end{equation*}

Una vez que tenemos esto podemos integrar cualquiera de las 3 respecto a la variable por la que fueron 
derivadas parcialmente para comenzar a construir la función $f(x,y,z)$.

\begin{equation*}
    \int f_{x}(x,y,z)dx=f(x,y,z)=xy^2+g(y,z)
\end{equation*}

Se usa el término $g(x,y)$ de manera temporal y será despejado conforme igualemos más derivadas parciales.\\

Primero derivamos $f(x,y,z)$ con respecto a $y$ para obtener $f_{y}(x,y,z)$ que es el mismo término que $Q$, a partir de esto
podremos despejar para $g_{y}(y,z)$ e integrar obteniendo así el valor de $g(y,z)$.
\begin{equation*}
    \begin{split}
        f_{y}(x,y,z)=2xy+g_{y}(y,z)&=2xy+e^{3z} \\
        \therefore g_{y}(y,z)&=e^{3z} \\
        \int g_{y}(y,z)dy&=\int e^{3z}dy\\
        g(y,z)&=ye^{3z}+h(z)+K\\
    \end{split}
\end{equation*}

Entonces, $f(x,y,z)$ queda en los siguientes términos actualizados.
\begin{equation*}
    f(x,y,z)=xy^2+ye^{3z}+h(z)+K
\end{equation*}

Luego de esto, derivamos $f(x,y,z)$ con respecto a $z$ y despejamos para encontrar $h'(z)$, seguido de esto integramos y obtenemos su valor.
\begin{equation*}
    \begin{split}
        f_{z}(x,y,z)=3ye^{3z}+h'(z)&=3ye^{3z}\\
        \therefore h'(z)&=0\\
        \int h'(z)dz&=\int 0dz\\
        h(z)&=L
    \end{split}
\end{equation*}

Ahora que tenemos todos los términos sustituimos y reducimos ambos independientes $K$ y $L$ a una sola constante $K$, obteniendo así nuestra función potencial $f(x,y,z)$.
\begin{equation*}
    f(x,y,z)=xy^2+ye^{3z}+K+L=xy^2+ye^{3z}+K
\end{equation*}
\end{document}
